\section{Data Structure and Memory Allocation}

This section describes the proposed data structure, found in the \cref{lst:bnode}. Comments and helpers methods were removed for the sake of brevity.

\begin{listing}[H]
  \begin{minted}{cpp}
template <typename KeyType, typename ValueType, size_t Order>
struct BNode {
  BNode * mSibling;
  KeyType mHighKey;

  uint16_t mHighKeyFlag;
  uint16_t mLeaf;
  uint16_t mSize;
  uint16_t mWriteLock;

  KeyType mKeys[Order];
  ValueType mValues[Order];
  BNode * mChildren[Order]; 
}
    \end{minted}
  \caption{The \code{BNode} struct}\label{lst:bnode}
\end{listing}

\mintinline{cpp}{uint16_t} have been chosen in favor of smaller data types, as most instructions in the \acrshort{isa} do not support operand types smaller than 16-bits and instead convert them to larger data types via a \mintinline{asm}{cvt} statement.

In the B-Link-Tree variant, \mintinline{cpp}{mSibling} and \mintinline{cpp}{mChildren} are using the \mintinline{cpp}{volatile} qualifier to avoid incorrect memory access optimization by the compiler and ensure the correctness of the tree during operations. This qualifier does incur a performance hit, however.

As the allocator for the data structure, a bump allocator is used (also known as stack allocator). The bump allocator allocates a continuous linear section of memory and works by increasing a pointer at the next unused memory. Atomic instructions are used to ensure the serializability of the operations, as the allocator is invoked concurrently from different threads.

To further reduce the number of instructions executed and cycles used, \mintinline{cpp}{asm} statement with specific PTX instructions is used instead of relying on compiler optimizations.