\section{B$^+$Tree}

\begin{definition}
  B$^+$Tree is a B-Tree where keys are stored exclusively in leaf nodes.
\end{definition}

Separators found in internal nodes can be freely chosen and may not match the actual keys in leaf nodes, as long as these separators split the tree into subtrees and preserve the ordering of the keys.

As the B$^+$Tree does not reuse the keys and may duplicate the keys found in the leaf nodes to use as separators in internal nodes, they do bear increased storage requirements. Compressing techniques on keys can be used to reduce the increased space complexity in exchange for increased complexity caused by compressing itself.

In most implementations, leaf nodes may include an additional pointer to a right sibling node, enabling straightforward sequential querying, which is helpful for range querying.

\begin{figure}[H]
  \centering
  \begin{figure}[H]
  \centering
  \begin{tikzpicture}[
    bnode/.style = {
        draw,
        rectangle split,
        rectangle split horizontal,
        rectangle split ignore empty parts,
        anchor=north
      },
    bchild/.style = { -> },
    bvalue/.style = { -{Circle[open]}, blue }
    ]

    % Level 0
    \node[bnode](y0) {
      \nodepart{one} $4$ \nodepart{two} $7$
    };

    % Level 1
    \draw[bchild](y0.south west) -- +(-4,-1) node[bnode](y1x0){
        \nodepart{one} $1$
      };

    \draw[bchild](y0.one split south) -- +(0,-1) node[bnode](y1x1){
        \nodepart{one} $1$ \nodepart{two} $2$
      };

    \draw[bchild](y0.south east) -- +(4,-1) node[bnode](y1x2){
        \nodepart{one} $1$
      };

    % Level 2
    \draw[bchild](y1x0.south west) -- +(-0.5,-1) node[bnode](y2x0){
        \nodepart{one} $1$
      };
    \draw[bvalue](y2x0.one south) -- +(0,-0.5);

    \draw[bchild](y1x0.south east) -- +(0.5,-1) node[bnode](y2x1){
        \nodepart{one} $1$ \nodepart{two} $2$
      };
    \draw[bvalue](y2x1.one south) -- +(0,-0.5);
    \draw[bvalue](y2x1.two south) -- +(0,-0.5);

    % ------
    \draw[bchild](y1x1.south west) -- +(-1.5,-1) node[bnode](y2x2){
        \nodepart{one} $1$
      };
    \draw[bvalue](y2x2.one south) -- +(0,-0.5);

    \draw[bchild](y1x1.one split south) -- +(0,-1) node[bnode](y2x3){
        \nodepart{one} $1$ \nodepart{two} $2$
      };
    \draw[bvalue](y2x3.one south) -- +(0,-0.5);
    \draw[bvalue](y2x3.two south) -- +(0,-0.5);

    \draw[bchild](y1x1.south east) -- +(1.5,-1) node[bnode](y2x4){
        \nodepart{one} $1$ \nodepart{two} $2$
      };
    \draw[bvalue](y2x4.one south) -- +(0,-0.5);
    \draw[bvalue](y2x4.two south) -- +(0,-0.5);

    % ------
    \draw[bchild](y1x2.south west) -- +(-0.5,-1) node[bnode](y2x5){
        \nodepart{one} $1$ \nodepart{two} $2$
      };
    \draw[bvalue](y2x5.one south) -- +(0,-0.5);
    \draw[bvalue](y2x5.two south) -- +(0,-0.5);

    \draw[bchild](y1x2.south east) -- +(0.5,-1) node[bnode](y2x6){
        \nodepart{one} $1$
      };
    \draw[bvalue](y2x6.one south) -- +(0,-0.5);

  \end{tikzpicture}
  \caption{B$^+$Tree with $\mathit{Order} = 3$}
\end{figure}

  \caption{B$^+$Tree with $\mathit{Order} = 3$.}
\end{figure}


All operations on the B$^+$Tree are simplified thanks to storing the keys exclusively in leaf nodes, as the end state when traversing a tree is always a leaf node.

Tree rebalancing operations in B$^+$Tree, such as splitting full nodes and merging free nodes, are the same as in B-Tree, with the only difference is the removal of the median key in leaf nodes. The median key is only propagated to the parent node after a split; the key itself is preserved in the leaf node, ensuring the rules of B$^+$Tree are not broken.