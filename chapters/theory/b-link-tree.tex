\section{B-Link-Tree}



Previous approaches include locking a subtree of highest affected node \cite{samadi1976b}, which, albeit straightforward, severely reduced concurrency.

In order to alleviate the bottleneck without risking inconsistency, \textit{B-Link-Tree} relaxes the definition of B$^+$Trees. As explained by Graefe \cite{goetz-tech}:

\begin{definition}
  B-Link-Tree is a B$^+$Tree with following properties:
  \begin{enumerate}
    \item Each node $x$ has additional attributes:
          \begin{itemize}
            \item $x.sibling$, a pointer to a right sibling node at the same depth,
            \item $x.highkey$, the maximum value found in the subtree rooted by $x$ (every key found in $x$ is less then $x.highkey$).
          \end{itemize}
    \item Does not require locks nor latches for reading.
  \end{enumerate}
\end{definition}

\begin{figure}[H]
  \centering
  \begin{tikzpicture}[
    bnode/.style = {
        draw,
        rectangle split,
        rectangle split horizontal,
        rectangle split ignore empty parts,
        anchor=north
      },
    bchild/.style = { -> },
    bsibling/.style = { ->, red },
    bvalue/.style = { -{Circle[open]}, blue }
    ]

    % Level 0
    \node[bnode](y0) {
      \nodepart{one} $4$ \nodepart{two} $7$
    };

    % Level 1
    \draw[bchild](y0.south west) -- +(-4,-1) node[bnode](y1x0){
        \nodepart{one} $1$
      };

    \draw[bchild](y0.one split south) -- +(0,-1) node[bnode](y1x1){
        \nodepart{one} $1$ \nodepart{two} $2$
      };

    \draw[bchild](y0.south east) -- +(4,-1) node[bnode](y1x2){
        \nodepart{one} $1$
      };

    \draw[bsibling](y1x0.east) -- (y1x1.west);
    \draw[bsibling](y1x1.east) -- (y1x2.west);

    % Level 2
    \draw[bchild](y1x0.south west) -- +(-0.5,-1) node[bnode](y2x0){
        \nodepart{one} $1$
      };
    \draw[bvalue](y2x0.one south) -- +(0,-0.5);

    \draw[bchild](y1x0.south east) -- +(0.5,-1) node[bnode](y2x1){
        \nodepart{one} $1$ \nodepart{two} $2$
      };
    \draw[bvalue](y2x1.one south) -- +(0,-0.5);
    \draw[bvalue](y2x1.two south) -- +(0,-0.5);

    % ------
    \draw[bchild](y1x1.south west) -- +(-1.5,-1) node[bnode](y2x2){
        \nodepart{one} $1$
      };
    \draw[bvalue](y2x2.one south) -- +(0,-0.5);

    \draw[bchild](y1x1.one split south) -- +(0,-1) node[bnode](y2x3){
        \nodepart{one} $1$ \nodepart{two} $2$
      };
    \draw[bvalue](y2x3.one south) -- +(0,-0.5);
    \draw[bvalue](y2x3.two south) -- +(0,-0.5);

    \draw[bchild](y1x1.south east) -- +(1.5,-1) node[bnode](y2x4){
        \nodepart{one} $1$ \nodepart{two} $2$
      };
    \draw[bvalue](y2x4.one south) -- +(0,-0.5);
    \draw[bvalue](y2x4.two south) -- +(0,-0.5);

    % ------
    \draw[bchild](y1x2.south west) -- +(-0.5,-1) node[bnode](y2x5){
        \nodepart{one} $1$ \nodepart{two} $2$
      };
    \draw[bvalue](y2x5.one south) -- +(0,-0.5);
    \draw[bvalue](y2x5.two south) -- +(0,-0.5);

    \draw[bchild](y1x2.south east) -- +(0.5,-1) node[bnode](y2x6){
        \nodepart{one} $1$
      };
    \draw[bvalue](y2x6.one south) -- +(0,-0.5);

    \draw[bsibling](y2x0.east) -- (y2x1.west);
    \draw[bsibling](y2x1.east) -- (y2x2.west);
    \draw[bsibling](y2x2.east) -- (y2x3.west);
    \draw[bsibling](y2x3.east) -- (y2x4.west);
    \draw[bsibling](y2x4.east) -- (y2x5.west);
    \draw[bsibling](y2x5.east) -- (y2x6.west);

  \end{tikzpicture}
  \caption{B-Link-Tree with $\mathit{Order} = 3$}
\end{figure}


Splitting during node insertion is divided into two independent steps: splitting the node and inserting a new separator key to the parent node.

\todo{Add explanation how high key is computed when splitting}

\todo{Add explanation how links are used during traversal}