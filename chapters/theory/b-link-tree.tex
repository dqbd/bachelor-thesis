\section{B-Link-Tree}

Previous approaches include locking a subtree of highest affected node \cite{samadi1976b}, which, albeit straightforward, severely reduced concurrency.

In order to alleviate the bottleneck without risking inconsistency, \textit{B-Link-Tree} relaxes the definition of B-Trees. As explained by Graefe \cite{goetz-tech}:

\begin{definition}[B-Link-Tree]
  B-Link-Tree divides node splitting into two independent steps: splitting the node and inserting a new separator key in the parent node. Each node thus has a pointer a sibling at the same level of the tree.
\end{definition}

\begin{figure}[H]
  \centering
  \begin{tikzpicture}[
      bnode/.style = {
          draw,
          text width=1em,
          align=center,
          rectangle split,
          rectangle split horizontal,
          rectangle split parts= 3,
        },
    ]
    \node[bnode](root) {
      \nodepart{one}    $1$
      \nodepart{two}    $2$
      \nodepart{three}  $3$
      \nodepart{four}   $4$
      \nodepart{five}
    };

    \draw[->](root.south west) -- +(-3,-1)
    node[bnode, anchor=north](a){
        \nodepart{one} $3$
        \nodepart{two} $9$
        \nodepart{three} $10$
        \nodepart{four} $12$
      };

    \draw[->](root.one split south) -- +(-1,-1)
    node[bnode, anchor=north](b) {
        \nodepart{one} $13$
        \nodepart{two} $14$
        \nodepart{three} $15$
        \nodepart{four}
      };

    \draw[->](a.east) -- (b.west);

    \draw[->](root.two split south) -- +(1,-1)
    node[bnode, anchor=north](c) {
        \nodepart{one} $17$
        \nodepart{two} $18$
        \nodepart{three}
        \nodepart{four}
      };

    \draw[->](b.east) -- (c.west);

    \draw[->](root.south east) -- +(3,-1)
    node[bnode, anchor=north](d) {
        \nodepart{one} $21$
        \nodepart{two} $29$
        \nodepart{three}
        \nodepart{four}
      };

    \draw[->](c.east) -- (d.west);

  \end{tikzpicture}
  \caption{B-Tree with $\mathit{Order} = 5$}
\end{figure}


\todo{Add explanation how high key is computed when splitting}

\todo{Add explanation how links are used during traversal}