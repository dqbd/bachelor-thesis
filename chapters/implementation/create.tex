\section{Bulk Loading}

As described by Graefe \cite{goetz-tech}, there is a strong relationship between B-Trees and sorting, which we can utilize. Optimal construction of B-Tree shall forgo incremental insertion in favor of building the tree from a presorted list.

As a B-Link-Tree internally stores its key-value pairs in nodes, we can avoid inserting these pairs individually. Instead, we construct the tree in a bottom-up approach, creating all nodes of a level at once, which can be seen in Algorithm \ref{alg:bulk-loading}.

\begin{algorithm}[H]
  \caption{Bulk Loading}\label{alg:bulk-loading}
  \SetKwInOut{Input}{Input}
  \SetKw{return}{return}
  \SetKwRepeat{Do}{do}{until}

  \textit{sortByKeyKernel(inputKeys, inputValues)}\\
  (keys, children) $\gets$ \textit{CreateLeafKernel(inputKeys, inputValues)}\\
  \While{\textit{children.len()} $> 1$}{
    (keys, children) $\gets$ \textit{CreateInternalKernel(keys, children)}
  }
  \textit{setRootNode(children[0])}
\end{algorithm}

We sort the key-value pairs with the \code{thrust::sort} function, as the TNL library does not have a sort function yet. Multiple kernel invocations are used to create appropriate nodes for each level.

The expected node count for each level is calculated by dividing the number of pairs by the desired node size. A fixed offset is chosen to make sure each node is not immediately full.

$$\mathit{nodeCount} = \ceil{\frac{|\mathit{inputKeys}|}{\mathit{Order} - \mathit{Offset}}},\quad \begin{aligned} \mathit{Order} > 0\end{aligned}$$


