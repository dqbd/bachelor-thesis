\section{Environment}

The measurements are captured on a machine with following specifications:

\begin{itemize}
  \item[CPU]{AMD Ryzen 5 1600 @ 3.200\,GHz \\ (6-core, 12-thread processor, 20\,032\,KiB total cache)}
  \item[GPU]{NVIDIA GeForce GTX 1650 Super, 3\,903\,MiB, Driver: 460.56}
  \item[RAM]{32\,GB (4$\times$8\,GB @ 2\,400\,MHz)}
  \item[OS]{Ubuntu 20.04.2 LTS x86\_64}
\end{itemize}

In terms of compiler configuration, \code{g++ 9.3.0} was used as the host compiler and \code{nvcc v11.0.211} was used as the device compiler.

\section{Testing methodology}

GTest, executing tests on different CTU FNSPE systems. Can we prove the implementation is error-proof? Not likely, as the execution is not deterministic. Both the internal state of the node and the results of the operations are tested.

\section{Benchmarking methodology}

After measuring execution times from chosen implementations, we calculate the speedup as the ratio of serial execution time on the CPU and parallel execution time on the GPU.

$$\mathit{Speedup} = \frac{\tau_{\mathit{CPU}}}{\tau_{\mathit{GPU}}}$$

What order of nodes is used for testing insertion/searching: increasing, decreasing, Knuth shuffled, gaussian, Zipf.

What timers are used? Host code uses \code{std::chrono::high\_resolution\_clock::now}, device based implementations are measured using CUDA events and CUDA runtime API (\code{cudaEvent*}). Time spent on copying inputs and outputs are not included. \todo{Is this a good idea?}

Increasing data sizes from $2^10$ until $2^25$ using 4-byte keys and values. Each measurement is captured 10 times and the results are averaged.

Synchronization does not seem to be an issue, memory access is key to improving performace. Randomized memory access, disordered memory access.

\todo{Add an explanation why \cite{awad} is 2-3 times faster than our implementation.} Improved memory access

\section{Comparison against known implementations}