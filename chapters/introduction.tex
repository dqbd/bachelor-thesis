\chapter{Introduction}\label{chapter:introduction}

\section{Motivation}

B-Tree is a well-known data structure often used as an index, a critical primitive used in various applications, such as data mining, decision support systems, and Online analytical processing (OLAP) \cite{olap,goetz-tech}. Widely found in database systems and file systems \cite{postgresql,btrfs}, B-Tree became the data structure of choice used when dealing with a large amount of arbitrary data of an unknown domain.

Thanks to the self-balancing nature of this data structure, by keeping the tree height as small as possible, regardless of arbitrary updates, lookup, insertion, and deletion in the tree are in $\mathcal{O}(\log{n})$. Naively, operating on multiple items will thus perform in $\mathcal{O}(n \log{n})$. This can be improved upon by using concurrent processing and computation on such data structure when processing multiple items.

One such hardware utilizing many processing cores to process tasks faster is the Graphical Processing Unit (GPU). With the advent of programmable shaders and support for floating-point operations, General-purpose computing on GPU became suitable for high-performance computing. Thanks to their superior computing performance to price ratio, GPUs became an essential staple for enabling high-performance computing for the masses.

Although GPUs have carved their niche and adopted some computational tasks, such as deep learning or cryptocurrency mining, an efficient GPU implementation of a B-Tree is challenging. Much attention needs to be paid when implementing concurrency control to allow high-throughput GPU performance without compromising correctness.

Nevertheless, a GPU-friendly implementation of such data structure can utilize the immense parallel processing power to accelerate querying and updating of multiple items at once. Such optimizations can unlock significant performance gains, benefiting all applications which use B-Trees.

\section{Structure of Work}

The main work of this thesis is to introduce an implementation of a B-Tree built on top of the Template Numerical Library (TNL). This implementation will have the ability to change the execution environment to either run on the GPU or the CPU. The solution will be implemented in \CC\ and will support CUDA, as these are the prerequisites of the used TNL library.

First, in \cref{chapter:preliminaries}, hardware structure and software architecture will be introduced, with a general primer of both the TNL library and the B-Tree data structure. Next, in \cref{chapter:theory}, different variants and modifications of the data structure, such as B$^+$Tree or B-Link-Tree, are explained and compared against the original B-Tree.

Previous implementations of such data structure and other state-of-the-art CPU solutions are discussed in \cref{chapter:state-of-art}. The realization section presented in \cref{chapter:realisation} will explain the design decision choices and various implementation and optimization details made for both of the GPU B-Tree variants shown in this thesis.

Last but not least, implementation correctness, testing methodology, and the experimental benchmark results between the developed solution and other chosen CPU and GPU implementations are presented in \cref{chapter:testing}.