\section{Prior Art}

Previous projects related to GPU implementation of B-trees focused on query throughput. Fix et al. \cite{fix2011accelerating} measured substantial performance speedup compared to sequential CPU execution by running independent queries in each thread block. Until recently, the general approach for updating is either to perform such updates on the CPU or to rebuild the structure from scratch, which is the case of Fix et al. implementation.

Kaczmarski \cite{kaczmarski} utilized the relationship between a sorted list and demonstrated improved insertion time by presorting keys and proposing a novel bottom-up approach when constructing a tree. The work also explored several methods of key searching within a tree node.

In Yan et al. \cite{harmonia}, instead of having the keys and children of a node in one instance, they are stored into separate arrays. Keys are stored in a one-dimensional array, and children are stored in a prefix-sum array. Storing keys and children into separate array improve memory locality and thus decreasing memory latency.

Awad et al. \cite{awad} proposed the most comprehensive GPU implementation of B-Tree to date. Their implementation supports concurrent queries (single, range, successor), insertion, and deletion. To achieve state-of-art performance, B-Link-Tree has been coupled with proactive splitting and Warp Cooperative Work Sharing as proposed by Ashkiani et al. \cite{ashkiani2018dynamic}.

