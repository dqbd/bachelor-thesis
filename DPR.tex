% překládejte pomocí příkazu xelatex
\documentclass{article}

\usepackage{graphicx}
\usepackage{url}

%-------- odkomentujte pokud chcete pro seznam literatury používat biblatex, číselné odkazy  -------
%\usepackage[style=iso-numeric]{biblatex}
%\addbibresource{mybibliography.bib}
%----------------------------------------------------------------------------------------------------

\title{Implementation of GPU B-tree} %doplňte název bakalářské práce
\author{
    \small Author: Tat Dat Duong\\
    \small Supervisor: \\
    \small Specialization:
} %doplňte své jméno, jméno vedoucího a svůj studijní obor
\date{\small \url{username@fit.cvut.cz}}

\begin{document}

\maketitle              

\paragraph{Keywords}{First keyword, Second keyword, Another keyword}
%doplňte cca 5 až 10 klíčových slov oddělených čárkou
%klíčová slova jsou odborné termíny popisující zaměření práce
%mohou být vyjádřeny i více slovy (např. "konečný automat")

\section{Introduction}
%popis problému, který práce řeší + lze popsat i co práce neřeší (pokud je to dle vás potřeba)
%důvody, proč je zajímavé/důležité se problémem zabývat, přínosy práce
%vaše motivace pro výběr tématu (je-li dle vás zajímavá)
%návaznost na jiné závěrečné práce 
%popis struktury práce (stručný popis následujících sekcí)

\section{Aim of the Thesis}
%stanovte si cíl(e) pro svou práci (naplnění pak zhodnotíte v závěru)
%lze rozdělit na hlavní cíle a dílčí (pokud máte více cílů)
The main goal of this bachelor thesis is to introduce a valid and working implementation of B-Tree data structure performing its' operation on GPU. 

Further goals include familiarizing with programming GPU cards on NVIDIA CUDA platform and extending Template Numerical Library with patterns and data structures to accommodate development of graph algorithms. 

We aim to measure performance speedup in comparison to reference state-of-the-art CPU implementation as well as competing GPU implementations of B-Tree and other data structures, such as hash table.

\section{Previous Work}
%teoretická část, popis definic a zavedení používaných pojmů/metod atd. 
%popis dosavadních poznatků (rešerše) k danému problému (kdo a jak problém řešil, k čemu došel, výhody a nevýhody) 
%tato kapitola prokazuje vaši znalost daného tématu, zde budete nejvíce citovat

% CUDA
% TNL
% B-Tree
% B-Link-Tree

\section{Your Approach}
%zde by měl být popsán váš přístup k řešení daného problému, odůvodnění zvolených technik, experimenty, výpočty, testování, popis implementace apod.
%jedná se o tvůrčí část práce, tj. musí být poznat, co je vaše práce

% Implementing CPU implementation
% Experiment with potential splitting techniques
% Add latching mechanism
% Add debugging tool
% Naively move into CUDA
% Utilize Warp Cooperative Work Scheduling work balance

\section{Conclusion}
%shrnutí cíle (cílů) práce a zhodnocení jeho (jejich) naplnění
%uvedení dosažených výsledků, komentář k využití daného řešení v praxi
%možné podněty pro navazující práce (výhled do budoucna)

% ---- Seznam literatury ----
% použít můžete prostředí "thebibliography" nebo "biblatex" -- (ne)výhody jednotlivých řešení budou probrány na cvičení

% THEBIBLIOGRAPHY -- smažte, pokud chcete pro seznam literatury používat biblatex
\begin{thebibliography}{9}
\bibitem{ukazka1} 
ŠESTÁKOVÁ, Eliška. 
\textit{Automaty a gramatiky: sbírka řešených příkladů}.
1.~vydání. Praha: České vysoké učení technické, 2017. ISBN 978-80-01-06306-4.
\end{thebibliography}

% BIBLATEX -- odkomentujte, pokud chcete pro seznam literatury používat biblatex
%\printbibliography

\end{document}