B-Tree je datová struktura, která provádí vkládání, mazání a vyhledávání klíčů a hodnot se složitostí $\mathcal{O}(\log{n})$. Tato práce se zabývá jejich studiem a implementací na kartách GPU. Byly naimplementovány dvě varianty B-Tree: B$^+$Tree a B-Link-Tree, obě patřičně upravené pro paralelní zpracování. Tyto varianty jsou popsané a implementované pro grafické karty NVIDIA v jazyce \CC\ s pomocí CUDA API a TNL knihovny. V práci se uvedena analýza existujících GPU i CPU řešení a jednotlivé úpravy a optimalizace provedené na výsledných strukturách. Všechny implementace jsou řádně otestované, změřené a porovnané s vybranými implementacemi dostupné pro GPU a CPU.